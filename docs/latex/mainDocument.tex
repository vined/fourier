% Kompiuterijos katedros šablonas
% Template of Department of Computer Science II
% Versija 1.0 2015 m. kovas [ March, 2015]

\documentclass[a4paper,12pt,fleqn]{article}
\input{allPacks}

\newtoggle{inLithuanian}
%If the report is in Lithuanian, it is set to true; otherwise, change to false
\settoggle{inLithuanian}{true}

%create file preface.tex for the preface text
%if preface is needed set to true
\newtoggle{needPreface}
\settoggle{needPreface}{false}

\newtoggle{signaturesOnTitlePage}
\settoggle{signaturesOnTitlePage}{false}


\input{macros}

\begin{document}
    % #1 -report type, #2 - title, #3-7 students, #8 - supervisor
    \depttitlepage{Antra užduotis}{Cooley–Tukey greitosios Furje transformacijos algoritmas}{Edvinas Naraveckas}
    {}{}{}{}% students 2-5
    {}

    \tableofcontents

    %Introduction section: label is sec:intro
    \sectionWithoutNumber{\keyWordIntroduction}{intro}
    Darbe taikomas greitosios Furje transformacijos metodas.
    Implementacija atlikta \textit{Haskell} kalba.
    Greitosios ir discrečiosios Furje transformacijų rezultatai patikrinti su Octave programos fft funkcijos rezultatais.
    Grafikai sugeneruoti naudojant \textit{Gnuplot} įrankį.

    %the main part
    \section{Greitosios ir diskrečiosios Furje transformacijų palyginimas}
    \label{sec:dftVsFft}

    Šiame palyginime buvo naudojmas tas pats audio signalas.
    Jam buvo pritaikoma transformacija ir tokio paties tipo atvirkštinė.
    Greitosio transformacijos algoritmas apdorodamas 96.2 kB audio signalą su 24025 reikšmių užtruko 28.5 sekundės.
    Tuo tarpu diskrečiojai Furje transformacijos versijai prireikė 6.6 minutės (398s).\cite{signals}


    \section{Dažnių filtravimas}
    \label{sec:filtering}




    %Conclusions section
    \sectionWithoutNumber{\keyWordConclusions}{conclusion}
    Greitoji Furje transformacija veikia žymiai greičiau už diskrečiąją.


    %file literatureSources.bib
    \referenceSources{literatureSources}


    %% this part is optional
    \newpage
    \begin{appendices}
        \section{Programos kodas}
        \label{app:code}
        \lstinputlisting[language=Haskell]{../../src/FastFourier.hs}
        \lstinputlisting[language=Haskell]{../../src/DiscreteFourier.hs}
    \end{appendices}

\end{document}
